\documentclass{report}
\usepackage[spanish]{babel}
\usepackage{enumerate}
\usepackage{amsmath} 
\usepackage{graphicx}
\usepackage[left=2.5cm, right=2.5cm, top=3cm, bottom=3cm]{geometry}

\title{Simulación de Gestión de Inventario en una Tienda}
\author{Ariadna Velázquez Rey     C31}
\date{ }

\begin{document}

\maketitle

\section*{S1. Introducción}

Este informe presenta una simulación basada en eventos discretos para analizar la gestión de inventario de una tienda, siguiendo el modelo del Ejemplo 7.6 del libro \textit{''Simulation''} de Sheldon M. Ross. El sistema simulado busca estimar la ganancia promedio por unidad de tiempo y determinar una política \((s, S)\) óptima para el manejo de inventario.

\subsection*{Objetivos}

\begin{itemize}
\item Estimar el beneficio esperado del sistema hasta un tiempo final predefinido \(T\).
\item Evaluar diferentes políticas de inventario \((s, S)\) y determinar cuál maximiza la ganancia.
\item Analizar la sensibilidad del sistema a variaciones en parámetros clave como la demanda, el costo de almacenamiento y la tasa de llegadas.
\end{itemize}

\subsection*{Variables del sistema}

\begin{itemize}
\item \(x\): Inventario disponible.
\item \(y\): Inventario pedido pero no recibido.
\item \(t\): Tiempo actual del sistema.
\item \(t_0\): Tiempo de llegada del próximo cliente.
\item \(t_1\): Tiempo de entrega del pedido en curso.
\item \(C\): Costos acumulados de pedidos.
\item \(H\): Costos acumulados de almacenamiento.
\item \(R\): Ingresos acumulados por ventas.
\end{itemize}


\section*{S2. Detalles de Implementación}

\subsection*{Descripción de la lógica del modelo}

\begin{itemize}
\item El sistema evoluciona en función de eventos: llegada de clientes (Poisson) y entregas de pedidos.
\item La demanda de cada cliente sigue una distribución geométrica desplazada: \(D \sim Geom(p=0.3)+1\).
\item Se sigue una política \((s, S)\): cuando el inventario \(x < s\) y \(y = 0\), se realiza un pedido de \(S - x\) unidades.
\item El sistema inicia en \(x = S\), sin pedidos en curso.
\end{itemize}

\subsection*{Algoritmo de simulación (resumen)}

\begin{enumerate}
\item Iniciar con \(t=0\), \(x=S\), \(y=0\).
\item Programar \(t_0\) como la próxima llegada de cliente.
\item Repetir hasta \(t \geq T\):
  \begin{itemize}
  \item Si \(t_0 < t_1\): procesar cliente, actualizar inventario y ventas.
  \item Si \(t_1 \leq t_0\): recibir pedido, actualizar inventario y costos.
  \end{itemize}
\item Calcular beneficio promedio: \[ \text{Ganancia Promedio} = \frac{R - C - H}{T} \]
\end{enumerate}

\subsection*{Implementación computacional}

\begin{itemize}
\item Lenguaje: Python
\item Librerías: NumPy, Pandas, Seaborn, Matplotlib
\item Se ejecutaron 1000 réplicas para estabilizar resultados.
\item Se exploraron 9 combinaciones de políticas \((s, S)\).
\end{itemize}


\section*{S3. Resultados y Experimentos}

\subsection*{Hallazgos principales}

\begin{itemize}
\item Para \((s=5, S=20)\) se obtuvo una ganancia promedio positiva con distribución simétrica.
\item La distribución de ganancias fue aproximadamente normal, lo cual valida la cantidad de réplicas.
\item El heatmap de políticas muestra que los mejores resultados se obtuvieron con \((s=5, S=25)\) y \((s=5, S=20)\).
\end{itemize}

\subsection*{Interpretación de resultados}

\begin{itemize}
\item Políticas con valores altos de \(S\) generan mayores ingresos pero también mayores costos de almacenamiento.
\item Políticas con valores bajos de \(s\) tienden a generar pedidos más frecuentes.
\end{itemize}

\subsection*{Validación y experimentos}

\begin{itemize}
\item Se comparó el rendimiento promedio de cada política \((s, S)\).
\item Se graficó la relación entre ingresos y costos totales para validar el comportamiento económico del sistema.
\end{itemize}

\subsection*{Variables de interés analizadas}

\begin{itemize}
\item Ganancia promedio
\item Costo total
\item Ingresos totales
\end{itemize}

\subsection*{Análisis de parada}

\begin{itemize}
\item Se detiene la simulación al superar un tiempo \(T = 30\).
\item El comportamiento de la ganancia promedio estabiliza luego de $\sim$800 réplicas.
\end{itemize}


\section*{S4. Modelo Matemático}

\subsection*{Definición del modelo}

\begin{itemize}
\item Llegadas: proceso Poisson con tasa \(\lambda = 2\).
\item Demanda: variable aleatoria geométrica desplazada: \(D \sim Geom(0.3)+1\).
\item Tiempo de entrega fijo: \(L = 2\).
\item Costos:
  \[ c(y) = 5y, \quad h = 0.5 \text{ por unidad-tiempo} \]
\item Ingreso por unidad vendida: \(r = 10\)
\end{itemize}

\subsection*{Supuestos}

\begin{itemize}
\item No hay backorders: demanda insatisfecha se pierde.
\item No hay costo fijo de ordenar, solo variable.
\item El sistema comienza lleno \(x = S\).
\end{itemize}

\subsection*{Comparación con resultados empíricos}

\begin{itemize}
\item La simulación reproduce el comportamiento esperado del modelo.
\item Las ganancias calculadas empíricamente concuerdan con el modelo teórico bajo distribuciones Poisson y geométrica.
\end{itemize}


\section*{S5. Conclusiones}

\begin{itemize}
\item El modelo basado en eventos discretos permite simular fielmente la dinámica de inventario.
\item La simulación es sensible a la política \((s, S)\) seleccionada, afectando costos, ingresos y nivel de servicio.
\item Se identificó \((s=5, S=25)\) como una de las mejores políticas bajo los parámetros actuales.
\item Se recomienda realizar análisis de sensibilidad para validar la robustez de la política seleccionada ante cambios en \(\lambda, h, r\) y distribución de demanda.
\item La herramienta de simulación puede extenderse para evaluar escenarios con demanda no estacionaria o tiempos de entrega aleatorios.
\end{itemize}


\end{document}


